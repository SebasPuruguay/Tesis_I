\chapter{METODOLOGÍA DE LA INVESTIGACIÓN}
\section{Diseño de la investigación}
En esta sección del documento se explicará cual es el diseño, el tipo y el enfoque del trabajo de
investigación, así como también la población y la muestra. 
%Para finalizar se explicará el
%proceso de aplicación de las redes neuronales convolucionales.
\subsection{Diseño no experimental}
El diseño es no experimental longitudinal, ya que las variables no serán manipuladas y serán analizadas tal como se encuentran. Es decir, tanto los datos textuales (noticias) y el precio del cobre serán analizados sin ningún cambio aplicando técnicas de procesamiento de lenguaje natural y algoritmos de aprendizaje automático con la finalidad de crear un modelo productivo robusto y facilitar la predicción del cobre. Asimismo, la recolección de datos que se realizará será en un determinado periodo de tiempo. 

\subsection{Tipo explicativo}
El alcance de la presente investigación es explicativo debido a que se busca explicar el comportamiento volátil del precio del cobre en base a noticias de periódicos digitales y además predecirlo.

\subsection{Enfoque cuantitativo}
El enfoque esta investigación es cuantitativo dado que se empleará técnicas del procesamiento de lenguaje natural (NLP), las cuales conllevan a procesar los datos de tipo textual a numéricos (vectores de características) y con ello posteriormente usar técnicas estadísticas como la regresión lineal para la predicción del precio del cobre.



\section{Población y muestra}

 Nisi porta lorem mollis aliquam ut porttitor leo. Aenean pharetra magna ac placerat vestibulum. Est placerat in egestas erat imperdiet sed euismod. Velit euismod in pellentesque massa placerat. Enim praesent elementum facilisis leo vel fringilla. Ante in nibh mauris cursus mattis molestie a iaculis. Erat pellentesque adipiscing commodo elit at imperdiet dui accumsan sit. Porttitor lacus luctus accumsan tortor posuere ac ut. Tortor at auctor urna nunc id. A iaculis at erat pellentesque adipiscing commodo elit. La Figura \ref{fig1} y el Cuadro \ref{tab:widgets}

	\begin{figure}[h]
		\begin{center}
			\includegraphics[width=0.8\textwidth]{3/figures/largepotato.jpg}
			\caption{Prueba de Figura}
			\label{fig1}
		\end{center}
		
	\end{figure}


\section{Operacionalización de Variables}

Nisi porta lorem mollis aliquam ut porttitor leo. Aenean pharetra magna ac placerat vestibulum. Est placerat in egestas erat imperdiet sed euismod. Velit euismod in pellentesque massa placerat. Enim praesent elementum facilisis leo vel fringilla. Ante in nibh mauris cursus mattis molestie a iaculis. Erat pellentesque adipiscing commodo elit at imperdiet dui accumsan sit. Porttitor lacus luctus accumsan tortor posuere ac ut. Tortor at auctor urna nunc id. A iaculis at erat pellentesque adipiscing commodo elit.
\section{Instrumentos de medida}
Nisi porta lorem mollis aliquam ut porttitor leo. Aenean pharetra magna ac placerat \begin{itemize}
	\item muscle and fat cells remove glucose from the blood,
	\item cells breakdown glucose via glycolysis and the citrate cycle, storing its energy in the form of ATP,
	\item liver and muscle store glucose as glycogen as a short-term energy reserve,
	\item adipose tissue stores glucose as fat for long-term energy reserve, and
	\item cells use glucose for protein synthesis.
\end{itemize}

\section{Técnicas de recolección de datos}
Nisi porta lorem mollis aliquam ut porttitor leo. Aenean pharetra magna ac placerat vestibulum. Est placerat in egestas erat imperdiet sed euismod. Velit euismod in pellentesque massa placerat. Enim praesent elementum facilisis leo vel fringilla. Ante in nibh mauris cursus mattis molestie a iaculis. Erat pellentesque adipiscing commodo elit at imperdiet dui accumsan sit. Porttitor lacus luctus accumsan tortor posuere ac ut. Tortor at auctor urna nunc id. A iaculis at erat pellentesque adipiscing commodo elit.

\LaTeX{} is great at typesetting mathematics. Let $X_1, X_2, \ldots, X_n$ be a sequence of independent and identically distributed random variables with
\begin{equation}
	S_n = \frac{X_1 + X_2 + \cdots + X_n}{n}
	= \frac{1}{n}\sum_{i}^{n} X_i
	\label{eq1}
\end{equation}

La Ecuación \ref{eq1} denote their mean. Then as $n$ approaches infinity, the random variables $$\sqrt{n}(S_n - \mu)$$ converge in distribution to a normal $\mathcal{N}(0, \sigma^2)$.

\section{Técnicas para el procesamiento y análisis de la información}
Nisi porta lorem mollis aliquam ut porttitor leo. Aenean pharetra magna ac placerat vestibulum. Est placerat in egestas erat imperdiet sed euismod. Velit euismod in pellentesque massa placerat. Enim praesent elementum facilisis leo vel fringilla. Ante in nibh mauris cursus mattis molestie a iaculis. Erat pellentesque adipiscing commodo elit at imperdiet dui accumsan sit. Porttitor lacus luctus accumsan tortor posuere ac ut. Tortor at auctor urna nunc id. A iaculis at erat pellentesque adipiscing commodo elit.

You can make lists with automatic numbering \dots

\begin{enumerate}
	\item Like this,
	\item and like this.
\end{enumerate}
\dots or bullet points \dots
\begin{itemize}
	\item Like this,
	\item and like this.
\end{itemize}


\section{Cronograma de actividades y presupuesto}
Nisi porta lorem mollis aliquam ut porttitor leo. Aenean pharetra magna ac placerat vestibulum. Est placerat in egestas erat imperdiet sed euismod. Velit euismod in pellentesque massa placerat. Enim praesent elementum facilisis leo vel fringilla. Ante in nibh mauris cursus mattis molestie a iaculis. Erat pellentesque adipiscing commodo elit at imperdiet dui accumsan sit. Porttitor lacus luctus accumsan tortor posuere ac ut. Tortor at auctor urna nunc id. A iaculis at erat pellentesque adipiscing commodo elit.

\begin{table}[h]
	\centering
	\begin{tabular}{l|r}
		Item & Quantity \\\hline
		Widgets & 42 \\
		Gadgets & 13
	\end{tabular}
	\caption{\label{tab:widgets}An example table.}
\end{table}