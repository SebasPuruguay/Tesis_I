%\thispagestyle{plain}
\begin{center}
	%\vspace*{1.5cm}
	{\large \bfseries  Agradecimiento y Dedicatoria}
\end{center}
\vspace{0.5cm}

Durante la inducción en la empresa en la cual realicé mis segundas prácticas pre-profesionales, se realizaron varias actividades, entre ellas, una que me marcó positivamente. Esta consistía en comparar los tiempos de llegada de un punto a otro de una persona corriendo. Se caracterizó porque quien asumió el reto tuvo presente en su mente las personas y las razones por las cuales todos los días lucha y son su principal fuente de motivación.

Por ello, quiero dedicar este gran esfuerzo personal de trabajo de tesis a quienes siempre han estado a mi lado en los mejores y peores momentos, aquellos críticos en que definen el destino. Mi amada hermana Clarisabel, mis queridos padres Augusto e Isabel, mi familia en especial mis abuelos; y mis pocos, pero verdaderos y leales amigos de la universidad, colegio y trabajo. Todos ellos han sido y son cada uno, piedra fundamental en el desarrollo de mi ser como persona y profesional, así como también seres con los cuales siempre comparto gratos momentos. Su presencia en mi vida no ha sido una suerte más sino parte de mi destino.
Asimismo, luchar por mis sueños y mi país, y pensar cada día en solidificar su planificación me motivan emocionalmente hasta en aquellos momentos en que parece haber imposibles.

Quiero concluir esta sección, muy especial para mí, agradeciendo también a mi alma máter, la Universidad Esan, y al Programa Nacional de Becas (Pronabec) por hacer que estos 5 años entre el 2015 y 2019 sean mágicos y muy fructíferos. Tuve la oportunidad no solo de incrementar y potenciar mis conocimientos en distintas áreas académicas sino también de aprender de excelentes profesionales como mis profesores, conocer grandes amigos dentro y fuera de su campus (desde el primer ciclo como cachimbo hasta el último ciclo, en el CADE Universitario 2019, estudiantes de diferentes universidades y otras partes del Perú), ponerme a prueba en el exterior (en el II Congreso Internacional de Investigación en Colombia en el año 2017, y en el Summer School of Machine Learning en Rusia en el año 2020 luego de obtener mi grado de bachiller) y formar parte de la gran familia UE.

Por todos ellos, simplemente gracias.
\newline

